\documentclass{beamer}
\setbeamertemplate{navigation symbols}{}
\usepackage[utf8]{inputenc}
\usepackage{beamerthemeshadow}
\usepackage{listings}
\usepackage{xcolor}
\usepackage{soul}
\usepackage{ragged2e}

%Empieza el documento
\begin{document}
% Definimos titulo, autor, fecha.
\title{TensorFlow}
\author{M.B.P.}
\date{\today}
\maketitle

\newpage
\begin{frame}\frametitle{Resumen}
Aca va el resumen del trabajo
\end{frame}

\newpage
\section{Introduccion}
\begin{frame}\frametitle{¿Qué es TensorFlow?} 
Se trata de una librería de código abierto desarrollada por Google
para facilitar el diseño, construcción y entrenamiento de sistemas
de aprendizaje profundo.
\end{frame}

\newpage
\begin{frame}\frametitle{¿Qué es un tensor?} 
Un tensor es una generalización de vectores y matrices
potencialmente altas dimensiones. Internamente, TensorFlow
representa tensores como matrices n-dimensiones de base
datatypes.\\
\vspace{5mm}
Un tf.Tensor tiene las siguientes propiedades:\\
\vspace{5mm}
• Un tipo de dato (float32, int32, o cadena, por ejemplo).\\
• Una forma (es decir, el número de dimensiones que tiene\\
\hspace{4mm}y el tamaño de cada dimensión).\\
\end{frame}

\newpage
\begin{frame}\frametitle{¿Como importar la librería?} 
Para importar la libreria de TensorFlow en cualquier proyecto se
usa el comando:

\begin{center}
\textbf{import tensorflow as tf}
\end{center}

\end{frame}

\newpage
\section{Basico}
\begin{frame}\frametitle{Tipo de Tensor}
Tensorflow trabaja con diferentes tipos de elementos denominados
tensor,estos deben ser declarados para su uso dentro de la seccion
del tensor.
\end{frame}

\newpage
\subsection{Tabla de tipos de Tensor}
\begin{frame}\frametitle{Tabla de tipos de Tensor}
\begin{tabular}{|l|c|}
\hline
Tipo & Descripción\\ \hline
tf.float16 & Punto flotante de precisión media de 16 bits\\ \hline
tf.float32 & Punto flotante de precisión simple de 32 bits\\ \hline
tf.float64 & Punto flotante de doble precisión de 64 bits\\ \hline
tf.bfloat16 & Punto flotante truncado de 16 bits\\ \hline
tf.complex64 & Complejo de precisión simple de 64 bits\\ \hline
tf.complex128 & Complejo de doble precisión de 128 bits\\ \hline
tf.int8 & Entero con signo de 8 bits\\ \hline
tf.uint8 & Entero sin signo de 8 bits\\ \hline
tf.uint16 & Entero sin signo de 16 bits\\ \hline
\end{tabular}
\end{frame}

\newpage
\begin{frame}\frametitle{Tabla de tipos de Tensor}
\begin{tabular}{|l|c|}
\hline
Tipo & Descripción\\ \hline
tf.int16 & Entero con signo de 16 bits\\ \hline
tf.int32 & Entero con signo de 32 bits\\ \hline
tf.int64 & Entero con signo de 64 bits\\ \hline
tf.bool & Booleano\\ \hline
tf.qint8 & Entero cuantificado de 8 bits con signo\\ \hline
tf.quint8 & Número entero sin signo de 8 bits cuantificado\\ \hline
tf.qint16 & Entero cuantificado de 16 bits con signo\\ \hline
tf.quint16 & Número entero sin signo de 16 bits cuantificado\\ \hline
tf.qint32 & Número entero con signo de 32 bits cuantificado\\ \hline
\end{tabular}
\end{frame}

\newpage
\begin{frame}\frametitle{Declaracion de variables}
En la sintaxi de python las variables se declaran de forma sencilla
como enteros o flotantes:\\
\vspace{5mm}
\begin{tabular}{l c}
x = 12 & esta variable sera entera.\\
y = 2.5 & esta variable sera flotante.\\
\end{tabular}\\
\vspace{5mm}
Cuando se pretende usar TensorFlow la declaracion de las variables
cambia teniendo una forma mas especial en la sintaxi:\\
\vspace{5mm}
\begin{tabular}{l}
x = tf.constant(3.0, dtype=tf.float32)\\
y = tf.constant(4.0)\\
\end{tabular}\\
\vspace{5mm}
En este caso ambas son reconocidas como tipo float pero
declaradas de diferente manera.
\end{frame}

\newpage
\subsection{Inputs and Readers}
\begin{frame}\frametitle{Inputs and Readers}
\textbf{Placeholders}\\
TensorFlow proporciona una operación de marcador de posición
que debe ser alimentado con los datos sobre la ejecución:\\
\vspace{5mm}
• tf.placeholder\\
• tf.placeholder\_with\_default\\
\vspace{5mm}
Para la alimentación de SparseTensors que son de tipo compuesto,
hay una función de confort:\\
\vspace{5mm}
• tf.sparse\_placeholder
\end{frame}

\newpage
\begin{frame}\frametitle{Inputs and Readers}
\textbf{Readers}\\
TensorFlow proporciona un conjunto de clases Reader para leer
formatos de datos:\\
\vspace{5mm}
•tf.ReaderBase\\
•tf.TextLineReader\\
•tf.WholeFileReader\\
•tf.IdentityReader\\
•tf.TFRecordReader\\
•tf.FixedLengthRecordReader\\
\end{frame}

\newpage
\begin{frame}\frametitle{Sintaxis Inputs}
Usando \textcolor{blue}{placeholder} de la siguiente manera:\\
placeholder(dtype, shape=None, name=None)\\
\vspace{2.5mm}
\textcolor{blue}{dtype} es el tipo de elemento del tensor que sera alimentado.\\
\vspace{2.5mm}
\textcolor{blue}{shape} es la forma del tensor para ser alimentado esta puede ser
opcional si no se especifica la forma esta se adatara con lo que esta alimentado.\\
\vspace{2.5mm}
\textcolor{blue}{name} nombre para la operacion igual es opcional.\\
\vspace{5mm}
\textcolor{red}{Importante}: Este tensor producirá un error si se evalúa. Su valor debe ser alimentado mediante el uso de \textcolor{blue}{feed\_dict}, argumentos opcionales como \textcolor{blue}{Session.run()}, \textcolor{blue}{Tensor.eval()}, o \textcolor{blue}{Operation.run()}.
\end{frame}

\newpage
\begin{frame}\frametitle{Sintaxis Inputs}
Usando \textcolor{blue}{placeholder\_with\_default} de la siguiente manera:\\
placeholder\_with\_default(input, shape, name=None)\\
\vspace{2.5mm}
\textcolor{blue}{input} un Tensor que es el valor predeterminado para producir cuando la salida no se alimenta.\\
\vspace{2.5mm}
\textcolor{blue}{shape} puedeser un \textcolor{blue}{tf.TensorShape} o lista de \textcolor{blue}{ints}. La forma (posiblemente parcial) del tensor.\\
\vspace{2.5mm}
\textcolor{blue}{name} un nombre para la operacion (opcional).
\end{frame}

\newpage
\begin{frame}\frametitle{Sintaxis Inputs}
Usando \textcolor{blue}{sparse\_placeholder} de la siguiente manera:\\
sparse\_placeholder(dtype, shape=None, name=None)\\
\vspace{2.5mm}
\textcolor{blue}{dtype} es el tipo de elemento del tensor que sera alimentado.\\
\vspace{2.5mm}
\textcolor{blue}{shape} la forma del tensor a alimentar (opcional). Si no se especifica la forma, puede alimentar un tensor disperso de cualquier forma.\\
\vspace{2.5mm}
\textcolor{blue}{name} nombre para la operacion igual es opcional.\\
\vspace{5mm}
\textcolor{red}{Importante}: Este tensor escaso producirá un error si se evalúa. Su valor debe alimentarse utilizando el argumento opcional \textcolor{blue}{feed\_dict} para \textcolor{blue}{Session.run ()}, \textcolor{blue}{Tensor.eval ()}, o \textcolor{blue}{Operation.run ()}.
\end{frame}

\newpage
\begin{frame}\frametitle{Sintaxis Readers}

\end{frame}

\newpage
\subsection{Operaciones Básicas}
\begin{frame}\frametitle{Comandos: Operaciones Básicas}
\begin{tabular}{|l|c|}
\hline
Comando & Accion\\ \hline
add() & Suma entre dos nodos ($x + y$)\\ \hline
subtract() & Resta entre dos nodos ($x - y$)\\ \hline
multiply() & Multiplicacion entre dos nodos ($x * y$)\\ \hline
div() & Divicion entre dos nodos ($\frac{x}{y}$)\\ \hline
square() & Determina el cuatrado de un nodo ($x^{2}$)\\ \hline
pow() & Calcula la potencia de un nodo ($x^{y}$)\\ \hline
sqrt() & Determina la raiz cuadrada de un nodo ($\sqrt{x}$)\\ \hline
\end{tabular}
\end{frame}

\newpage
\begin{frame}\frametitle{Sintaxis}
Para el uso de cada comando es neceario entender su sitaxis y los
elementos que la componen en cada caso estan compuestos de tres
elementos pero puede variar dependiendo de la funcion:\\
\vspace{5mm}
Un elemento que se encuentra presente en estos comandos es el
\textcolor{blue}{name} el cual da un nombre para la operacion a realizar (este es
opcional y no afecta al uso de la misma).
\end{frame}

\newpage
\begin{frame}\frametitle{Sintaxis}
\begin{footnotesize}
Para la suma con el \textcolor{blue}{add} seria:\\
add(x, y, name=None)\\ \vspace{2.5mm}
• x,y seran un tensor del mismo tipo, los cual pueden ser uno de los siguientes
tipos: \textcolor{blue}{half, float32, float64, uint8, int8, int16, int32, int64, complex64,
complex128, string}\\
El resultado sera del mismo tipo ocupado en dicha función.\\ \vspace{5mm}
Para la resta usando \textcolor{blue}{subtract} seria:\\
subtract(x, y, name=None)\\ \vspace{2.5mm}
• x,y seran un tensor del mismo tipo, los cual pueden ser uno de los siguientes
tipos: \textcolor{blue}{half, float32, float64, uint8, int8, uint16, int16, int32, int64, complex64,
complex128}\\
El resultado sera del mismo tipo ocupado en dicha función.
\end{footnotesize}
\end{frame}

\newpage
\begin{frame}\frametitle{Sintaxis}
\begin{footnotesize}
Para la multiplicación usando \textcolor{blue}{multiply} tenemos:\\
multiply(x, y, name=None)\\ \vspace{2.5mm}
• x,y seran un tensor del mismo tipo, los cual pueden ser uno de los siguientes
tipos: \textcolor{blue}{half, float32, float64, uint8, int8, uint16, int16, int32, int64, complex64,
complex128}\\
El resultado sera del mismo tipo ocupado en dicha función.\\ \vspace{5mm}
Para la división usando \textcolor{blue}{div} tenemos:\\
div(x, y, name=None)\\ \vspace{2.5mm}
• x,y seran un tensor del mismo tipo,el tipo de numerador y denominado deben
ser numeros reales.\\
El resultado debera ser el cociente de x,y.
\end{footnotesize}
\end{frame}

\newpage
\begin{frame}\frametitle{Sintaxis}
\begin{footnotesize}
Para elevar un elemento al cuadrado se usa \textcolor{blue}{square} tenemos:\\
square(x, name=None)\\ \vspace{2.5mm}
• x puede ser un tensor o un sparsetensor,el tipo que se puede ocupar es: \textcolor{blue}{half,
float32, float64, int32, int64, complex64, complex128}\\
El resultado debe ser del mismo tipo que x.\\ \vspace{5mm}
Para elevar un elemento en cualquier exponente tenemos \textcolor{blue}{pow}:\\
pow(x, y, name=None)\\ \vspace{2.5mm}
• x,y deben ser tensor del mismo tipo, estos tipos son: \textcolor{blue}{float32, float64, int32,
int64, complex64, or complex128}\\
El resultado debe ser del mismo tipo del tensor.
\end{footnotesize}
\end{frame}

\newpage
\begin{frame}\frametitle{Sintaxis}
Para obtener la raízcuadrada de un elemento se usa \textcolor{blue}{sqrt}:\\
sqrt(x, name=None)\\ \vspace{2.5mm}
• x es un tensor o un sparsetensor, el tipo de este puede ser: \textcolor{blue}{half,
float32, float64, complex64, complex128}\\
El resultado debe ser un tensor o un sparsetensor del mismo tipo
ocupado.
\end{frame}

\newpage
\begin{footnotesize}
\begin{lstlisting}[language=Python]

import tensorflow as tf

x = tf.placeholder("float")
y = tf.placeholder("float")

suma = tf.add(x, y)

sess = tf.Session()

print ('suam=', sess.run(suma, feed_dict={a: 3, b: 3}))

\end{lstlisting}
\end{footnotesize}

\newpage
\section{c3}
Aqui empieza el capitulo sobre estado del arte
% Termina el documento
\end{document}